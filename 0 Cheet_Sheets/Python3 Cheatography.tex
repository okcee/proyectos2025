\documentclass[10pt,a4paper]{article}

% Packages
\usepackage{fancyhdr}           % For header and footer
\usepackage{multicol}           % Allows multicols in tables
\usepackage{tabularx}           % Intelligent column widths
\usepackage{tabulary}           % Used in header and footer
\usepackage{hhline}             % Border under tables
\usepackage{graphicx}           % For images
\usepackage{xcolor}             % For hex colours
%\usepackage[utf8x]{inputenc}    % For unicode character support
\usepackage[T1]{fontenc}        % Without this we get weird character replacements
\usepackage{colortbl}           % For coloured tables
\usepackage{setspace}           % For line height
\usepackage{lastpage}           % Needed for total page number
\usepackage{seqsplit}           % Splits long words.
%\usepackage{opensans}          % Can't make this work so far. Shame. Would be lovely.
\usepackage[normalem]{ulem}     % For underlining links
% Most of the following are not required for the majority
% of cheat sheets but are needed for some symbol support.
\usepackage{amsmath}            % Symbols
\usepackage{MnSymbol}           % Symbols
\usepackage{wasysym}            % Symbols
%\usepackage[english,german,french,spanish,italian]{babel}              % Languages

% Document Info
\author{Ismael Mercado (torerohk)}
\pdfinfo{
  /Title (python-3-espanol.pdf)
  /Creator (Cheatography)
  /Author (Ismael Mercado (torerohk))
  /Subject (Python 3 Español Cheat Sheet)
}

% Lengths and widths
\addtolength{\textwidth}{6cm}
\addtolength{\textheight}{-1cm}
\addtolength{\hoffset}{-3cm}
\addtolength{\voffset}{-2cm}
\setlength{\tabcolsep}{0.2cm} % Space between columns
\setlength{\headsep}{-12pt} % Reduce space between header and content
\setlength{\headheight}{85pt} % If less, LaTeX automatically increases it
\renewcommand{\footrulewidth}{0pt} % Remove footer line
\renewcommand{\headrulewidth}{0pt} % Remove header line
\renewcommand{\seqinsert}{\ifmmode\allowbreak\else\-\fi} % Hyphens in seqsplit
% This two commands together give roughly
% the right line height in the tables
\renewcommand{\arraystretch}{1.3}
\onehalfspacing

% Commands
\newcommand{\SetRowColor}[1]{\noalign{\gdef\RowColorName{#1}}\rowcolor{\RowColorName}} % Shortcut for row colour
\newcommand{\mymulticolumn}[3]{\multicolumn{#1}{>{\columncolor{\RowColorName}}#2}{#3}} % For coloured multi-cols
\newcolumntype{x}[1]{>{\raggedright}p{#1}} % New column types for ragged-right paragraph columns
\newcommand{\tn}{\tabularnewline} % Required as custom column type in use

% Font and Colours
\definecolor{HeadBackground}{HTML}{333333}
\definecolor{FootBackground}{HTML}{666666}
\definecolor{TextColor}{HTML}{333333}
\definecolor{DarkBackground}{HTML}{162542}
\definecolor{LightBackground}{HTML}{F7F8F9}
\renewcommand{\familydefault}{\sfdefault}
\color{TextColor}

% Header and Footer
\pagestyle{fancy}
\fancyhead{} % Set header to blank
\fancyfoot{} % Set footer to blank
\fancyhead[L]{
\noindent
\begin{multicols}{3}
\begin{tabulary}{5.8cm}{C}
    \SetRowColor{DarkBackground}
    \vspace{-7pt}
    {\parbox{\dimexpr\textwidth-2\fboxsep\relax}{\noindent
        \hspace*{-6pt}\includegraphics[width=5.8cm]{/web/www.cheatography.com/public/images/cheatography_logo.pdf}}
    }
\end{tabulary}
\columnbreak
\begin{tabulary}{11cm}{L}
    \vspace{-2pt}\large{\bf{\textcolor{DarkBackground}{\textrm{Python 3 Español Cheat Sheet}}}} \\
    \normalsize{by \textcolor{DarkBackground}{Ismael Mercado (torerohk)} via \textcolor{DarkBackground}{\uline{cheatography.com/23626/cs/5397/}}}
\end{tabulary}
\end{multicols}}

\fancyfoot[L]{ \footnotesize
\noindent
\begin{multicols}{3}
\begin{tabulary}{5.8cm}{LL}
  \SetRowColor{FootBackground}
  \mymulticolumn{2}{p{5.377cm}}{\bf\textcolor{white}{Cheatographer}}  \\
  \vspace{-2pt}Ismael Mercado (torerohk) \\
  \uline{cheatography.com/torerohk} \\
  \end{tabulary}
\vfill
\columnbreak
\begin{tabulary}{5.8cm}{L}
  \SetRowColor{FootBackground}
  \mymulticolumn{1}{p{5.377cm}}{\bf\textcolor{white}{Cheat Sheet}}  \\
   \vspace{-2pt}Published 12th November, 2015.\\
   Updated 11th May, 2016.\\
   Page {\thepage} of \pageref{LastPage}.
\end{tabulary}
\vfill
\columnbreak
\begin{tabulary}{5.8cm}{L}
  \SetRowColor{FootBackground}
  \mymulticolumn{1}{p{5.377cm}}{\bf\textcolor{white}{Sponsor}}  \\
  \SetRowColor{white}
  \vspace{-5pt}
  %\includegraphics[width=48px,height=48px]{dave.jpeg}
  Measure your website readability!\\
  www.readability-score.com
\end{tabulary}
\end{multicols}}




\begin{document}
\raggedright
\raggedcolumns

% Set font size to small. Switch to any value
% from this page to resize cheat sheet text:
% www.emerson.emory.edu/services/latex/latex_169.html
\footnotesize % Small font.

\begin{multicols*}{2}

\begin{tabularx}{8.4cm}{X}
\SetRowColor{DarkBackground}
\mymulticolumn{1}{x{8.4cm}}{\bf\textcolor{white}{print()}}  \tn
\SetRowColor{LightBackground}
\mymulticolumn{1}{x{8.4cm}}{print()} \tn 
\hhline{>{\arrayrulecolor{DarkBackground}}-}
\SetRowColor{LightBackground}
\mymulticolumn{1}{x{8.4cm}}{print al ser funcion siempre se utiliza con parentesis}  \tn 
\hhline{>{\arrayrulecolor{DarkBackground}}-}
\end{tabularx}
\par\addvspace{1.3em}

\begin{tabularx}{8.4cm}{X}
\SetRowColor{DarkBackground}
\mymulticolumn{1}{x{8.4cm}}{\bf\textcolor{white}{Variable String}}  \tn
\SetRowColor{LightBackground}
\mymulticolumn{1}{x{8.4cm}}{y = "a" \newline z = 'Hola' \newline  \newline multilinea = """cadena de texto \newline con mas de una linea"""} \tn 
\hhline{>{\arrayrulecolor{DarkBackground}}-}
\SetRowColor{LightBackground}
\mymulticolumn{1}{x{8.4cm}}{variables de tipo string van entre comillas dobles o sencillas \newline para cadenas de texto de multiples lineas se utiliza """ texto """}  \tn 
\hhline{>{\arrayrulecolor{DarkBackground}}-}
\end{tabularx}
\par\addvspace{1.3em}

\begin{tabularx}{8.4cm}{x{5.68 cm} p{2.32 cm} }
\SetRowColor{DarkBackground}
\mymulticolumn{2}{x{8.4cm}}{\bf\textcolor{white}{Operadores Matematicos}}  \tn
% Row 0
\SetRowColor{LightBackground}
suma & `a+b` \tn 
% Row Count 1 (+ 1)
% Row 1
\SetRowColor{white}
resta & `a-b` \tn 
% Row Count 2 (+ 1)
% Row 2
\SetRowColor{LightBackground}
multiplicacion & `a*b` \tn 
% Row Count 3 (+ 1)
% Row 3
\SetRowColor{white}
division\_real & `a/b` \tn 
% Row Count 4 (+ 1)
% Row 4
\SetRowColor{LightBackground}
division\_entera & `a//b` \tn 
% Row Count 5 (+ 1)
% Row 5
\SetRowColor{white}
resto & `a\%b` \tn 
% Row Count 6 (+ 1)
% Row 6
\SetRowColor{LightBackground}
potencia & `a**b` \tn 
% Row Count 7 (+ 1)
\hhline{>{\arrayrulecolor{DarkBackground}}--}
\SetRowColor{LightBackground}
\mymulticolumn{2}{x{8.4cm}}{los operadores matemáticos principales pueden utilizarse combinados respetando la jerarquía al resolverlas \newline 1. Resolver ( )  {[} {]}  \{ \} \newline 2. Resolver exponentes. \newline 3. Resolver * y / de izquierda a derecha \newline 4. Resolver  + y - de izquierda a derecha}  \tn 
\hhline{>{\arrayrulecolor{DarkBackground}}--}
\end{tabularx}
\par\addvspace{1.3em}

\begin{tabularx}{8.4cm}{x{6.48 cm} p{1.52 cm} }
\SetRowColor{DarkBackground}
\mymulticolumn{2}{x{8.4cm}}{\bf\textcolor{white}{Operadores Logicos}}  \tn
% Row 0
\SetRowColor{LightBackground}
Igual a & `==` \tn 
% Row Count 1 (+ 1)
% Row 1
\SetRowColor{white}
Diferente a & `!=` \tn 
% Row Count 2 (+ 1)
% Row 2
\SetRowColor{LightBackground}
Menor que & `\textless{}` \tn 
% Row Count 3 (+ 1)
% Row 3
\SetRowColor{white}
Menor o igual que & `\textless{}=` \tn 
% Row Count 4 (+ 1)
% Row 4
\SetRowColor{LightBackground}
Mayor que & `\textgreater{}` \tn 
% Row Count 5 (+ 1)
% Row 5
\SetRowColor{white}
Mayor o igual que & `\textgreater{}=` \tn 
% Row Count 6 (+ 1)
\hhline{>{\arrayrulecolor{DarkBackground}}--}
\SetRowColor{LightBackground}
\mymulticolumn{2}{x{8.4cm}}{Devolverán un valor boleano}  \tn 
\hhline{>{\arrayrulecolor{DarkBackground}}--}
\end{tabularx}
\par\addvspace{1.3em}

\begin{tabularx}{8.4cm}{X}
\SetRowColor{DarkBackground}
\mymulticolumn{1}{x{8.4cm}}{\bf\textcolor{white}{Metodos para Strings}}  \tn
\SetRowColor{white}
\mymulticolumn{1}{x{8.4cm}}{{\bf{len()}} retorna longitud de caracteres en string: \newline % Row Count 2 (+ 2)
`len(string)` \newline % Row Count 3 (+ 1)
{\bf{lower()}} retorna string en minúsculas: \newline % Row Count 4 (+ 1)
`string.lower()` \newline % Row Count 5 (+ 1)
{\bf{upper()}} retorna string en mayúsculas: \newline % Row Count 6 (+ 1)
`string.upper()` \newline % Row Count 7 (+ 1)
{\bf{capitalize()}} retorna primer carácter de string en mayúsculas \newline % Row Count 9 (+ 2)
`string.capitalize()` \newline % Row Count 10 (+ 1)
{\bf{str()}} retorna conversión explícita de strings: \newline % Row Count 12 (+ 2)
`str(string)`% Row Count 13 (+ 1)
} \tn 
\hhline{>{\arrayrulecolor{DarkBackground}}-}
\SetRowColor{LightBackground}
\mymulticolumn{1}{x{8.4cm}}{Literales \newline `variable.lower()` \newline `variable.upper()` \newline No Literales \newline `len(variable)` \newline `str(variable)`}  \tn 
\hhline{>{\arrayrulecolor{DarkBackground}}-}
\end{tabularx}
\par\addvspace{1.3em}

\begin{tabularx}{8.4cm}{X}
\SetRowColor{DarkBackground}
\mymulticolumn{1}{x{8.4cm}}{\bf\textcolor{white}{String Inmutable a Flexible}}  \tn
\SetRowColor{LightBackground}
\mymulticolumn{1}{x{8.4cm}}{print("\%s" \% (variable)) o print("\%s" \% ("string")) \newline  \newline nom= "Ismael" \newline ape = "Mercado" \newline  \newline \# variables \newline print ("mi nombre \%s. mi apellido \%s ." \% (nom, ape)) \newline  \newline \# strings \newline print ("mi nombre \%s. mi apellido \%s ." \% ("Ismael", "Mercado"))} \tn 
\hhline{>{\arrayrulecolor{DarkBackground}}-}
\end{tabularx}
\par\addvspace{1.3em}

\begin{tabularx}{8.4cm}{p{2.088 cm} x{2.304 cm} p{2.088 cm} p{0.72 cm} }
\SetRowColor{DarkBackground}
\mymulticolumn{4}{x{8.4cm}}{\bf\textcolor{white}{Comparadores guia}}  \tn
% Row 0
\SetRowColor{LightBackground}
 & {\bf{AND}} &  &  \tn 
% Row Count 1 (+ 1)
% Row 1
\SetRowColor{white}
True & True & True &  \tn 
% Row Count 2 (+ 1)
% Row 2
\SetRowColor{LightBackground}
True & False & False &  \tn 
% Row Count 3 (+ 1)
% Row 3
\SetRowColor{white}
False & True & False &  \tn 
% Row Count 4 (+ 1)
% Row 4
\SetRowColor{LightBackground}
False & False & False &  \tn 
% Row Count 5 (+ 1)
% Row 5
\SetRowColor{white}
 & {\bf{OR}} &  &  \tn 
% Row Count 6 (+ 1)
% Row 6
\SetRowColor{LightBackground}
True & True & True &  \tn 
% Row Count 7 (+ 1)
% Row 7
\SetRowColor{white}
True & False & True &  \tn 
% Row Count 8 (+ 1)
% Row 8
\SetRowColor{LightBackground}
False & True & True &  \tn 
% Row Count 9 (+ 1)
% Row 9
\SetRowColor{white}
False & False & False &  \tn 
% Row Count 10 (+ 1)
% Row 10
\SetRowColor{LightBackground}
 & {\bf{NOT}} &  &  \tn 
% Row Count 11 (+ 1)
% Row 11
\SetRowColor{white}
True &  & False &  \tn 
% Row Count 12 (+ 1)
% Row 12
\SetRowColor{LightBackground}
False &  & True &  \tn 
% Row Count 13 (+ 1)
\hhline{>{\arrayrulecolor{DarkBackground}}----}
\end{tabularx}
\par\addvspace{1.3em}

\begin{tabularx}{8.4cm}{X}
\SetRowColor{DarkBackground}
\mymulticolumn{1}{x{8.4cm}}{\bf\textcolor{white}{Diccionarios}}  \tn
\SetRowColor{white}
\mymulticolumn{1}{x{8.4cm}}{Estructura de datos que almacena valores utilizando otros como referencia para su acceso y almacenamiento, es iterable, mutable y puede contener elementos de diferente tipo; se declara entre llaves {\emph{\{clave:valor\}}} \newline % Row Count 5 (+ 5)
`diccionario=\{'a':1, 'b':2, 'c':3\}` \newline % Row Count 6 (+ 1)
Podemos utilizar la funcion {\bf{dict()}} \newline % Row Count 7 (+ 1)
`diccionario=dict(a=1, b=2, c=3)` \newline % Row Count 8 (+ 1)
Acceder a un elemento utilizamos el indice \newline % Row Count 9 (+ 1)
`diccionario{[}'c'{]}` \newline % Row Count 10 (+ 1)
Modificar un valor \newline % Row Count 11 (+ 1)
`diccionario{[}'b'{]}=28` \newline % Row Count 12 (+ 1)
Nuevos elementos añadimos una clave no existente \newline % Row Count 13 (+ 1)
`diccionario{[}'d'{]}=4` \newline % Row Count 14 (+ 1)
Iterar con un diccionario \newline % Row Count 15 (+ 1)
{\bf{items()}} {\emph{Acceso a claves y valores}} \newline % Row Count 16 (+ 1)
`diccionario.items()` \newline % Row Count 17 (+ 1)
{\bf{values()}} {\emph{Acceso a valores}} \newline % Row Count 18 (+ 1)
`diccionario.values()` \newline % Row Count 19 (+ 1)
{\bf{keys()}} {\emph{Acceso a claves}} \newline % Row Count 20 (+ 1)
`diccionario.keys()` \newline % Row Count 21 (+ 1)
Ordenar un diccionario \newline % Row Count 22 (+ 1)
`sorted(diccionario)` \newline % Row Count 23 (+ 1)
Ordenar un diccionario en inverso \newline % Row Count 24 (+ 1)
`sorted(diccionario, reverse=True)`% Row Count 25 (+ 1)
} \tn 
\hhline{>{\arrayrulecolor{DarkBackground}}-}
\end{tabularx}
\par\addvspace{1.3em}

\begin{tabularx}{8.4cm}{X}
\SetRowColor{DarkBackground}
\mymulticolumn{1}{x{8.4cm}}{\bf\textcolor{white}{Matrices}}  \tn
\SetRowColor{white}
\mymulticolumn{1}{x{8.4cm}}{Anidando listas construimos matrices de elementos \newline % Row Count 1 (+ 1)
`matriz={[}{[}1,2,3{]},{[}4,5,6{]}{]}` \newline % Row Count 2 (+ 1)
para acceder a los elementos utilizamos \newline % Row Count 3 (+ 1)
`matriz{[}0{]}{[}1{]}` \newline % Row Count 4 (+ 1)
sustituir un elemento \newline % Row Count 5 (+ 1)
`matriz{[}1{]}{[}0{]}=33`% Row Count 6 (+ 1)
} \tn 
\hhline{>{\arrayrulecolor{DarkBackground}}-}
\end{tabularx}
\par\addvspace{1.3em}

\begin{tabularx}{8.4cm}{X}
\SetRowColor{DarkBackground}
\mymulticolumn{1}{x{8.4cm}}{\bf\textcolor{white}{crear, modificar y leer archivos en disco}}  \tn
\SetRowColor{white}
\mymulticolumn{1}{x{8.4cm}}{Función para crear un archivo \newline % Row Count 1 (+ 1)
`def crearArchivo():` \newline % Row Count 2 (+ 1)
\seqsplit{`   archivo=open('datos}.txt', 'w')` \newline % Row Count 3 (+ 1)
\seqsplit{`    archivo.close()`} \newline % Row Count 4 (+ 1)
Función para escribir en un archivo \newline % Row Count 5 (+ 1)
`def escribirArchivo():` \newline % Row Count 6 (+ 1)
\seqsplit{`    archivo=open('datos}.txt', 'a')` \newline % Row Count 7 (+ 1)
\seqsplit{`    archivo.write('prueba} de texto\textbackslash{}n')` \newline % Row Count 8 (+ 1)
`    archivo.close` \newline % Row Count 9 (+ 1)
Función para leer un archivo \newline % Row Count 10 (+ 1)
`def leerArchivo():` \newline % Row Count 11 (+ 1)
\seqsplit{`    archivo=open('datos}.txt', 'r')` \newline % Row Count 12 (+ 1)
`    linea = archivo.readline()` \newline % Row Count 13 (+ 1)
`    while linea!="":` \newline % Row Count 14 (+ 1)
\seqsplit{`        print(linea)`} \newline % Row Count 15 (+ 1)
\seqsplit{`        linea=archivo}.readline()` \newline % Row Count 16 (+ 1)
\seqsplit{`    archivo.close()`}% Row Count 17 (+ 1)
} \tn 
\hhline{>{\arrayrulecolor{DarkBackground}}-}
\end{tabularx}
\par\addvspace{1.3em}

\begin{tabularx}{8.4cm}{x{1.064 cm} x{3.268 cm} x{3.268 cm} }
\SetRowColor{DarkBackground}
\mymulticolumn{3}{x{8.4cm}}{\bf\textcolor{white}{Modos de apertura de archivos}}  \tn
% Row 0
\SetRowColor{LightBackground}
{\bf{Indicador}} & {\bf{Modo de apertura}} & {\bf{Ubicación del puntero}} \tn 
% Row Count 3 (+ 3)
% Row 1
\SetRowColor{white}
``r` & Solo lectura & \textasciicircum{}Al inicio del archivo\textasciicircum{} \tn 
% Row Count 5 (+ 2)
% Row 2
\SetRowColor{LightBackground}
`rb` & Solo lectura en modo binario & \textasciicircum{}Al inicio del archivo\textasciicircum{} \tn 
% Row Count 7 (+ 2)
% Row 3
\SetRowColor{white}
`r+` & Lectura y escritura & \textasciicircum{}Al inicio del archivo\textasciicircum{} \tn 
% Row Count 9 (+ 2)
% Row 4
\SetRowColor{LightBackground}
`rb+` & Lectura y escritura en modo binario & \textasciicircum{}Al inicio del archivo\textasciicircum{} \tn 
% Row Count 12 (+ 3)
% Row 5
\SetRowColor{white}
`w` & Solo escritura. Sobreescribe el archivo si existe. Crea el archivo si no existe & \textasciicircum{}Al inicio del archivo\textasciicircum{} \tn 
% Row Count 17 (+ 5)
% Row 6
\SetRowColor{LightBackground}
`wb` & Solo escritura en modo binario. Sobreescribe el archivo si existe. Crea el archivo si no existe & \textasciicircum{}Al inicio del archivo\textasciicircum{} \tn 
% Row Count 23 (+ 6)
% Row 7
\SetRowColor{white}
`w+` & Escritura y lectura. Sobreescribe el archivo si existe. Crea el archivo si no existe & \textasciicircum{}Al inicio del archivo\textasciicircum{} \tn 
% Row Count 28 (+ 5)
% Row 8
\SetRowColor{LightBackground}
`wb+` & Escritura y lectura en modo binario. Sobreescribe el archivo si existe. Crea el archivo si no existe & \textasciicircum{}Al inicio del archivo\textasciicircum{} \tn 
% Row Count 34 (+ 6)
\end{tabularx}
\par\addvspace{1.3em}

\vfill
\columnbreak
\begin{tabularx}{8.4cm}{x{1.064 cm} x{3.268 cm} x{3.268 cm} }
\SetRowColor{DarkBackground}
\mymulticolumn{3}{x{8.4cm}}{\bf\textcolor{white}{Modos de apertura de archivos (cont)}}  \tn
% Row 9
\SetRowColor{LightBackground}
`a` & Añadido (agregar contenido). Crea el archivo si éste no existe & \textasciicircum{}Si archivo existe, al final. Si no, al comienzo\textasciicircum{} \tn 
% Row Count 4 (+ 4)
% Row 10
\SetRowColor{white}
`ab` & Añadido en modo binario (agregar contenido). Crea el archivo si éste no existe & \textasciicircum{}Si archivo existe, al final. Si no, al comienzo\textasciicircum{} \tn 
% Row Count 9 (+ 5)
% Row 11
\SetRowColor{LightBackground}
`a+` & Añadido (agregar contenido) y lectura. Crea el archivo si éste no existe. & \textasciicircum{}Si archivo existe, al final. Si no, al comienzo\textasciicircum{} \tn 
% Row Count 14 (+ 5)
% Row 12
\SetRowColor{white}
`ab+` & Añadido (agregar contenido) y lectura en modo binario. Crea el archivo si éste no existe & \textasciicircum{}Si archivo existe, al final. Si no, al comienzo\textasciicircum{} \tn 
% Row Count 20 (+ 6)
\hhline{>{\arrayrulecolor{DarkBackground}}---}
\SetRowColor{LightBackground}
\mymulticolumn{3}{x{8.4cm}}{indicado a la función open() como una string en su segundo parámetro.}  \tn 
\hhline{>{\arrayrulecolor{DarkBackground}}---}
\end{tabularx}
\par\addvspace{1.3em}

\begin{tabularx}{8.4cm}{x{2.736 cm} x{2.356 cm} x{2.508 cm} }
\SetRowColor{DarkBackground}
\mymulticolumn{3}{x{8.4cm}}{\bf\textcolor{white}{Funciones integradas}}  \tn
% Row 0
\SetRowColor{LightBackground}
\seqsplit{`\_\_import\_\_()`} & `abs()` & `all()` \tn 
% Row Count 1 (+ 1)
% Row 1
\SetRowColor{white}
`any()` & `ascii()` & `bin()` \tn 
% Row Count 2 (+ 1)
% Row 2
\SetRowColor{LightBackground}
`bool()` & \seqsplit{`bytearray()`} & `bytes()` \tn 
% Row Count 4 (+ 2)
% Row 3
\SetRowColor{white}
`callable()` & `chr()` & \seqsplit{`classmethod()`} \tn 
% Row Count 6 (+ 2)
% Row 4
\SetRowColor{LightBackground}
`compile()` & `complex()` & `delattr()` \tn 
% Row Count 7 (+ 1)
% Row 5
\SetRowColor{white}
`dict()` & `dir()` & `divmod()` \tn 
% Row Count 8 (+ 1)
% Row 6
\SetRowColor{LightBackground}
`enumerate()` & `eval()` & `exec()` \tn 
% Row Count 9 (+ 1)
% Row 7
\SetRowColor{white}
`filter()` & `float()` & `format()` \tn 
% Row Count 10 (+ 1)
% Row 8
\SetRowColor{LightBackground}
`frozenset()` & `getattr()` & `globals()` \tn 
% Row Count 11 (+ 1)
% Row 9
\SetRowColor{white}
`hasattr()` & `hash()` & `help()` \tn 
% Row Count 12 (+ 1)
% Row 10
\SetRowColor{LightBackground}
`hex()` & `id()` & `input()` \tn 
% Row Count 13 (+ 1)
% Row 11
\SetRowColor{white}
`int()` & \seqsplit{`isinstance()`} & \seqsplit{`issubclass()`} \tn 
% Row Count 15 (+ 2)
% Row 12
\SetRowColor{LightBackground}
`iter()` & `len()` & `list()` \tn 
% Row Count 16 (+ 1)
% Row 13
\SetRowColor{white}
`locals()` & `map()` & `max()` \tn 
% Row Count 17 (+ 1)
% Row 14
\SetRowColor{LightBackground}
\seqsplit{`memoryview()`} & `min()` & `next()` \tn 
% Row Count 18 (+ 1)
% Row 15
\SetRowColor{white}
`object()` & `oct()` & `open()` \tn 
% Row Count 19 (+ 1)
% Row 16
\SetRowColor{LightBackground}
`ord()` & `pow()` & `print()` \tn 
% Row Count 20 (+ 1)
% Row 17
\SetRowColor{white}
`property()` & `range()` & `repr()` \tn 
% Row Count 21 (+ 1)
% Row 18
\SetRowColor{LightBackground}
`reversed()` & `round()` & `set()` \tn 
% Row Count 22 (+ 1)
% Row 19
\SetRowColor{white}
`setattr()` & `slice()` & `sorted()` \tn 
% Row Count 23 (+ 1)
% Row 20
\SetRowColor{LightBackground}
\seqsplit{`staticmethod()`} & `str()` & `sum()` \tn 
% Row Count 25 (+ 2)
% Row 21
\SetRowColor{white}
`super()` & `tuple()` & `type()` \tn 
% Row Count 26 (+ 1)
% Row 22
\SetRowColor{LightBackground}
`vars()` & `zip()` &  \tn 
% Row Count 27 (+ 1)
\hhline{>{\arrayrulecolor{DarkBackground}}---}
\SetRowColor{LightBackground}
\mymulticolumn{3}{x{8.4cm}}{Python incluye las siguientes funciones y siempre están disponibles}  \tn 
\hhline{>{\arrayrulecolor{DarkBackground}}---}
\end{tabularx}
\par\addvspace{1.3em}

\begin{tabularx}{8.4cm}{X}
\SetRowColor{DarkBackground}
\mymulticolumn{1}{x{8.4cm}}{\bf\textcolor{white}{type()}}  \tn
\SetRowColor{LightBackground}
\mymulticolumn{1}{x{8.4cm}}{x = 3.1415 \newline print(type(x)) \newline  \newline \textgreater{}\textgreater{}\textless{}class 'float'\textgreater{}} \tn 
\hhline{>{\arrayrulecolor{DarkBackground}}-}
\SetRowColor{LightBackground}
\mymulticolumn{1}{x{8.4cm}}{La función type permite comprobar el tipo de variable}  \tn 
\hhline{>{\arrayrulecolor{DarkBackground}}-}
\end{tabularx}
\par\addvspace{1.3em}

\begin{tabularx}{8.4cm}{X}
\SetRowColor{DarkBackground}
\mymulticolumn{1}{x{8.4cm}}{\bf\textcolor{white}{Variables Numericas}}  \tn
\SetRowColor{LightBackground}
\mymulticolumn{1}{x{8.4cm}}{num\_entero = 5 \newline num\_negativo = -7 \newline num\_real = 3.14 \newline num\_complejo = 3.2 + 7j \newline num\_binario = 0b111 \newline num\_octal = 0o10 \newline num\_hex = 0xff} \tn 
\hhline{>{\arrayrulecolor{DarkBackground}}-}
\SetRowColor{LightBackground}
\mymulticolumn{1}{x{8.4cm}}{puedes crear variables del tipo Enteros, Reales, Complejos y los puedes representar en Decimal, Binario, Octal y Hexadecimal}  \tn 
\hhline{>{\arrayrulecolor{DarkBackground}}-}
\end{tabularx}
\par\addvspace{1.3em}

\begin{tabularx}{8.4cm}{X}
\SetRowColor{DarkBackground}
\mymulticolumn{1}{x{8.4cm}}{\bf\textcolor{white}{Conjunto Matematico funcion set()}}  \tn
\SetRowColor{LightBackground}
\mymulticolumn{1}{x{8.4cm}}{conjunto = set('246') \newline conjunto2 = \{2, 4, 6\}} \tn 
\hhline{>{\arrayrulecolor{DarkBackground}}-}
\SetRowColor{LightBackground}
\mymulticolumn{1}{x{8.4cm}}{se pueden utilizar los métodos {\emph{add()}} y {\emph{remove()}} para añadir o eliminar elementos. \newline si se crea un conjunto con valores repetidos, estos se eliminan automáticamente.}  \tn 
\hhline{>{\arrayrulecolor{DarkBackground}}-}
\end{tabularx}
\par\addvspace{1.3em}

\begin{tabularx}{8.4cm}{x{1.2 cm} x{6.8 cm} }
\SetRowColor{DarkBackground}
\mymulticolumn{2}{x{8.4cm}}{\bf\textcolor{white}{Operadores Comparadores}}  \tn
% Row 0
\SetRowColor{LightBackground}
{\bf{and}} & compara 2 elementos y devuelve True si ambos son verdaderos \tn 
% Row Count 2 (+ 2)
% Row 1
\SetRowColor{white}
{\bf{or}} & compara 2 elementos y devuelve True si uno de ellos es verdadero \tn 
% Row Count 4 (+ 2)
% Row 2
\SetRowColor{LightBackground}
{\bf{not}} & devuelve el valor opuesto de un boleano \tn 
% Row Count 6 (+ 2)
\hhline{>{\arrayrulecolor{DarkBackground}}--}
\SetRowColor{LightBackground}
\mymulticolumn{2}{x{8.4cm}}{primero se calcula {\emph{not}} \newline después se calcula {\emph{and}} \newline por último se calcula {\emph{or}}}  \tn 
\hhline{>{\arrayrulecolor{DarkBackground}}--}
\end{tabularx}
\par\addvspace{1.3em}

\begin{tabularx}{8.4cm}{x{1.6 cm} x{6.4 cm} }
\SetRowColor{DarkBackground}
\mymulticolumn{2}{x{8.4cm}}{\bf\textcolor{white}{Definiciones}}  \tn
% Row 0
\SetRowColor{LightBackground}
\seqsplit{Iteración} & Término general para la toma de cada elemento de algo, una después de la otra. Usar un bucle, explícita o implícita, al pasar sobre un grupo de elementos \tn 
% Row Count 5 (+ 5)
\hhline{>{\arrayrulecolor{DarkBackground}}--}
\end{tabularx}
\par\addvspace{1.3em}

\begin{tabularx}{8.4cm}{X}
\SetRowColor{DarkBackground}
\mymulticolumn{1}{x{8.4cm}}{\bf\textcolor{white}{Metodos Especiales para Strings}}  \tn
\SetRowColor{white}
\mymulticolumn{1}{x{8.4cm}}{{\bf{find()}} Retorna el indice del primer carácter que coincide con el buscado \newline % Row Count 2 (+ 2)
`cad = "ABC"` \newline % Row Count 3 (+ 1)
`cad.find("B")` \newline % Row Count 4 (+ 1)
`\textgreater{}\textgreater{}1` \newline % Row Count 5 (+ 1)
{\bf{replace()}} reemplaza un carácter por otro \newline % Row Count 6 (+ 1)
`cad.replace("B", "Z")` \newline % Row Count 7 (+ 1)
`\textgreater{}\textgreater{}AZC` \newline % Row Count 8 (+ 1)
{\bf{split()}} divide una cadena basado en un caracter y retorna una lista \newline % Row Count 10 (+ 2)
`cad.split(";")` \newline % Row Count 11 (+ 1)
{\bf{join()}} retorna una cadena donde los valores son separados por un caracter \newline % Row Count 13 (+ 2)
`lista = {[}"Hola", "Mundo"{]}` \newline % Row Count 14 (+ 1)
`print ("+".join(lista,))` \newline % Row Count 15 (+ 1)
`lista2 = "Hola"` \newline % Row Count 16 (+ 1)
`print ("-".join(lista2))` \newline % Row Count 17 (+ 1)
{\bf{strip()}}, {\bf{lstrip()}}, {\bf{rstrip()}} eliminan los espacios en blanco, a la izquierda y a la derecha respectivamente \newline % Row Count 20 (+ 3)
`cad.strip()` \newline % Row Count 21 (+ 1)
`cad.lstrip()` \newline % Row Count 22 (+ 1)
`cad.rstrip()`% Row Count 23 (+ 1)
} \tn 
\hhline{>{\arrayrulecolor{DarkBackground}}-}
\end{tabularx}
\par\addvspace{1.3em}

\begin{tabularx}{8.4cm}{x{3.344 cm} p{1.52 cm} x{2.736 cm} }
\SetRowColor{DarkBackground}
\mymulticolumn{3}{x{8.4cm}}{\bf\textcolor{white}{Tabla Basica}}  \tn
% Row 0
\SetRowColor{LightBackground}
Tupla & `( )` & Inmutable \tn 
% Row Count 1 (+ 1)
% Row 1
\SetRowColor{white}
Lista & `{[} {]}` & Mutable \tn 
% Row Count 2 (+ 1)
% Row 2
\SetRowColor{LightBackground}
Diccionario & `\{ \}` & Mutable \tn 
% Row Count 3 (+ 1)
\hhline{>{\arrayrulecolor{DarkBackground}}---}
\end{tabularx}
\par\addvspace{1.3em}

\begin{tabularx}{8.4cm}{X}
\SetRowColor{DarkBackground}
\mymulticolumn{1}{x{8.4cm}}{\bf\textcolor{white}{Tupla}}  \tn
\SetRowColor{white}
\mymulticolumn{1}{x{8.4cm}}{Arreglo de objetos definido entre paréntesis es inmutable puede contener diferentes tipos de objetos. \newline % Row Count 3 (+ 3)
`tupla = (1, 'a', 3.5)` \newline % Row Count 4 (+ 1)
Se puede anidar una tupla dentro de otra \newline % Row Count 5 (+ 1)
`tupla2 = (1, (4, 'B'), 3.5)` \newline % Row Count 6 (+ 1)
Se puede acceder a los valores a través del indice. \newline % Row Count 8 (+ 2)
`tupla{[}1{]}`% Row Count 9 (+ 1)
} \tn 
\hhline{>{\arrayrulecolor{DarkBackground}}-}
\end{tabularx}
\par\addvspace{1.3em}

\begin{tabularx}{8.4cm}{X}
\SetRowColor{DarkBackground}
\mymulticolumn{1}{x{8.4cm}}{\bf\textcolor{white}{Lista}}  \tn
\SetRowColor{white}
\mymulticolumn{1}{x{8.4cm}}{Arreglo de objetos definido entre corchetes es mutable puede contener diferentes tipos de objetos. \newline % Row Count 2 (+ 2)
`lista = {[}2, 'B', 4.5{]}` \newline % Row Count 3 (+ 1)
Se puede acceder a los valores a través del indice y reemplazarlos. \newline % Row Count 5 (+ 2)
`lista{[}1{]} = 'A'` \newline % Row Count 6 (+ 1)
Podemos comprobar si un valor existe en una lista usando {\bf{in}}. \newline % Row Count 8 (+ 2)
`'B' in lista` \newline % Row Count 9 (+ 1)
se insertan valores al final de la lista con {\bf{.append()}} \newline % Row Count 11 (+ 2)
`lista.append('nuevo')` \newline % Row Count 12 (+ 1)
insertar en una posición definida se utiliza el indice y {\bf{.insert()}} \newline % Row Count 14 (+ 2)
`lista.insert(2, 'C')` \newline % Row Count 15 (+ 1)
borrar un elemento usamos {\bf{del()}} \newline % Row Count 16 (+ 1)
`del(lista{[}1{]})` \newline % Row Count 17 (+ 1)
ordenar sin alterar {\bf{sorted()}} y para orden inverso argumento {\bf{reverse}} \newline % Row Count 19 (+ 2)
`sorted(lista)` \newline % Row Count 20 (+ 1)
`sorted(lista, reverse=True)` \newline % Row Count 21 (+ 1)
ordenar con criterio como argumento \newline % Row Count 22 (+ 1)
`sorted(lista, key=str.lower)` \newline % Row Count 23 (+ 1)
ordenar alterando usamos {\bf{sort()}} \newline % Row Count 24 (+ 1)
`lista.sort()`% Row Count 25 (+ 1)
} \tn 
\hhline{>{\arrayrulecolor{DarkBackground}}-}
\end{tabularx}
\par\addvspace{1.3em}

\begin{tabularx}{8.4cm}{X}
\SetRowColor{DarkBackground}
\mymulticolumn{1}{x{8.4cm}}{\bf\textcolor{white}{Comprensión de Listas y Diccionarios}}  \tn
\SetRowColor{white}
\mymulticolumn{1}{x{8.4cm}}{Compresión Lista \newline % Row Count 1 (+ 1)
`lista= {[}x for x in (1,2,3){]}` \newline % Row Count 2 (+ 1)
Compresión Diccionario \newline % Row Count 3 (+ 1)
`diccionario= \{k: k+1 for k in (1,2,3)\}`% Row Count 4 (+ 1)
} \tn 
\hhline{>{\arrayrulecolor{DarkBackground}}-}
\SetRowColor{LightBackground}
\mymulticolumn{1}{x{8.4cm}}{La comprensión es una construcción sintáctica de python, permite declarar una lista o diccionario a través de la creación de otra.}  \tn 
\hhline{>{\arrayrulecolor{DarkBackground}}-}
\end{tabularx}
\par\addvspace{1.3em}

\begin{tabularx}{8.4cm}{X}
\SetRowColor{DarkBackground}
\mymulticolumn{1}{x{8.4cm}}{\bf\textcolor{white}{For y While}}  \tn
\SetRowColor{white}
\mymulticolumn{1}{x{8.4cm}}{El bucle {\bf{while}} (mientras) ejecuta un fragmento de código mientras se cumpla una condición. \newline % Row Count 2 (+ 2)
`edad = 0` \newline % Row Count 3 (+ 1)
`while edad \textless{} 18:` \newline % Row Count 4 (+ 1)
`    edad = edad + 1` \newline % Row Count 5 (+ 1)
`    print "Felicidades, tienes " + str(edad)`% Row Count 7 (+ 2)
} \tn 
\hhline{>{\arrayrulecolor{DarkBackground}}-}
\SetRowColor{LightBackground}
\mymulticolumn{1}{x{8.4cm}}{Permiten ejecutar un mismo fragmento de código un cierto número de veces, mientras se cumpla una determinada condición.}  \tn 
\hhline{>{\arrayrulecolor{DarkBackground}}-}
\end{tabularx}
\par\addvspace{1.3em}

\begin{tabularx}{8.4cm}{X}
\SetRowColor{DarkBackground}
\mymulticolumn{1}{x{8.4cm}}{\bf\textcolor{white}{If, Else y Elif}}  \tn
\SetRowColor{white}
\mymulticolumn{1}{x{8.4cm}}{Evalúan la condición indicada y ejecutan una instrucción u otra \newline % Row Count 2 (+ 2)
`if condicion1:` \newline % Row Count 3 (+ 1)
{\emph{si condicion1 es True realiza esto}} \newline % Row Count 4 (+ 1)
`elif condicion2:` \newline % Row Count 5 (+ 1)
{\emph{si condicion2 es True realiza esto}} \newline % Row Count 6 (+ 1)
`else:` \newline % Row Count 7 (+ 1)
{\emph{si ambas condiciones son False realiza esto}} \newline % Row Count 8 (+ 1)
se pueden anidar \newline % Row Count 9 (+ 1)
`if condicion1:` \newline % Row Count 10 (+ 1)
{\emph{si condicion1 es True realiza esto}} \newline % Row Count 11 (+ 1)
`    if condicion3:` \newline % Row Count 12 (+ 1)
{\emph{    si condicion3 es True realiza esto}} \newline % Row Count 13 (+ 1)
`    else:` \newline % Row Count 14 (+ 1)
{\emph{    si es False realiza esto}} \newline % Row Count 15 (+ 1)
`else:` \newline % Row Count 16 (+ 1)
{\emph{en caso contrario realiza esto}}% Row Count 17 (+ 1)
} \tn 
\hhline{>{\arrayrulecolor{DarkBackground}}-}
\end{tabularx}
\par\addvspace{1.3em}

\begin{tabularx}{8.4cm}{X}
\SetRowColor{DarkBackground}
\mymulticolumn{1}{x{8.4cm}}{\bf\textcolor{white}{Clases, Objetos, Propiedades y Metodos}}  \tn
\SetRowColor{white}
\mymulticolumn{1}{x{8.4cm}}{*Clases \newline % Row Count 1 (+ 1)
 *Objetos \newline % Row Count 2 (+ 1)
  *Propiedades \newline % Row Count 3 (+ 1)
   *Métodos \newline % Row Count 4 (+ 1)
`class Clase():` \# La clase \newline % Row Count 5 (+ 1)
`    varClase=0` \# Variables de Clase \newline % Row Count 6 (+ 1)
`    def \_\_init\_\_(self):` \# Método de Instancia (constructor) \newline % Row Count 8 (+ 2)
\seqsplit{`        self.varInstancia=0`} \# Variable de Instancia \newline % Row Count 10 (+ 2)
`objeto=Clase()` \newline % Row Count 11 (+ 1)
\seqsplit{`objeto.metodoinstancia()`} \newline % Row Count 12 (+ 1)
`@classmethod` \# Decorador Metodo de Clase \newline % Row Count 13 (+ 1)
`    def clsmet(cls):` \# Obligatorio (cls) \newline % Row Count 14 (+ 1)
`Clase.clsmet`% Row Count 15 (+ 1)
} \tn 
\hhline{>{\arrayrulecolor{DarkBackground}}-}
\SetRowColor{LightBackground}
\mymulticolumn{1}{x{8.4cm}}{{\bf{Self}} hace referencia a si mismo \newline {\bf{\_\_init\_\_}} constructor para inicializar los objetos a un valor \newline al colocar {\emph{(clase)}} se habilita la hereda los objetos de la clase Persona}  \tn 
\hhline{>{\arrayrulecolor{DarkBackground}}-}
\end{tabularx}
\par\addvspace{1.3em}

\begin{tabularx}{8.4cm}{x{2.356 cm} x{2.66 cm} x{2.584 cm} }
\SetRowColor{DarkBackground}
\mymulticolumn{3}{x{8.4cm}}{\bf\textcolor{white}{Palabras reservadas}}  \tn
% Row 0
\SetRowColor{LightBackground}
`and` & `as` & `assert` \tn 
% Row Count 1 (+ 1)
% Row 1
\SetRowColor{white}
`break` & `class` & `continue` \tn 
% Row Count 2 (+ 1)
% Row 2
\SetRowColor{LightBackground}
`def` & `del` & `elif` \tn 
% Row Count 3 (+ 1)
% Row 3
\SetRowColor{white}
`else` & `except` & `False` \tn 
% Row Count 4 (+ 1)
% Row 4
\SetRowColor{LightBackground}
`finally` & `for` & `from` \tn 
% Row Count 5 (+ 1)
% Row 5
\SetRowColor{white}
`global` & `if` & `import` \tn 
% Row Count 6 (+ 1)
% Row 6
\SetRowColor{LightBackground}
`in` & `is` & `lambda` \tn 
% Row Count 7 (+ 1)
% Row 7
\SetRowColor{white}
`None` & `nonlocal` & `not` \tn 
% Row Count 8 (+ 1)
% Row 8
\SetRowColor{LightBackground}
`or` & `pass` & `raise` \tn 
% Row Count 9 (+ 1)
% Row 9
\SetRowColor{white}
`return` & `True` & `try` \tn 
% Row Count 10 (+ 1)
% Row 10
\SetRowColor{LightBackground}
`while` & `with` & `yield` \tn 
% Row Count 11 (+ 1)
\hhline{>{\arrayrulecolor{DarkBackground}}---}
\SetRowColor{LightBackground}
\mymulticolumn{3}{x{8.4cm}}{Estas palabras no pueden utilizarse para nombrar variables.}  \tn 
\hhline{>{\arrayrulecolor{DarkBackground}}---}
\end{tabularx}
\par\addvspace{1.3em}

\begin{tabularx}{8.4cm}{x{2.32 cm} x{5.68 cm} }
\SetRowColor{DarkBackground}
\mymulticolumn{2}{x{8.4cm}}{\bf\textcolor{white}{Patrones caracteres}}  \tn
% Row 0
\SetRowColor{LightBackground}
`\textbackslash{}n` & Nueva Linea \tn 
% Row Count 1 (+ 1)
% Row 1
\SetRowColor{white}
`\textbackslash{}r` & Retorno de carro \tn 
% Row Count 2 (+ 1)
% Row 2
\SetRowColor{LightBackground}
`\textbackslash{}t` & Tabulador Horizontal \tn 
% Row Count 3 (+ 1)
% Row 3
\SetRowColor{white}
`\textbackslash{}w` & Caracter minuscula \tn 
% Row Count 4 (+ 1)
% Row 4
\SetRowColor{LightBackground}
`\textbackslash{}W` & Caracter Mayuscula \tn 
% Row Count 5 (+ 1)
% Row 5
\SetRowColor{white}
`\textbackslash{}s` & Engloba minusculas y mayusculas \tn 
% Row Count 7 (+ 2)
% Row 6
\SetRowColor{LightBackground}
`\textbackslash{}S` & cualquier caracter que no es espacio en blanco \tn 
% Row Count 9 (+ 2)
% Row 7
\SetRowColor{white}
`\textbackslash{}d` & numero entre 0 - 9 \tn 
% Row Count 10 (+ 1)
% Row 8
\SetRowColor{LightBackground}
`\textbackslash{}D` & cualquier carácter que no es un numero \tn 
% Row Count 12 (+ 2)
% Row 9
\SetRowColor{white}
`\textasciicircum{}` & Inicio de cadena \tn 
% Row Count 13 (+ 1)
% Row 10
\SetRowColor{LightBackground}
`\$` & Fin de cadena \tn 
% Row Count 14 (+ 1)
% Row 11
\SetRowColor{white}
`\textbackslash{}` & Escape caracter especial \tn 
% Row Count 15 (+ 1)
% Row 12
\SetRowColor{LightBackground}
`{[}{]}` & rango de caracteres dentro de corchetes \tn 
% Row Count 17 (+ 2)
% Row 13
\SetRowColor{white}
`\textasciicircum{}{[}{]}` & cualquier caracter fuera de corchetes \tn 
% Row Count 19 (+ 2)
% Row 14
\SetRowColor{LightBackground}
`\textbackslash{}b` & separacion entre numero y/o letra \tn 
% Row Count 21 (+ 2)
% Row 15
\SetRowColor{white}
\{\{Metacaracter\}\} & repeticiones \tn 
% Row Count 23 (+ 2)
% Row 16
\SetRowColor{LightBackground}
`+` & una o mas veces \tn 
% Row Count 24 (+ 1)
% Row 17
\SetRowColor{white}
`*` & cero o mas veces \tn 
% Row Count 25 (+ 1)
% Row 18
\SetRowColor{LightBackground}
`?` & cero o una vez \tn 
% Row Count 26 (+ 1)
% Row 19
\SetRowColor{white}
`\{n\}` & n numero de veces \tn 
% Row Count 27 (+ 1)
\hhline{>{\arrayrulecolor{DarkBackground}}--}
\end{tabularx}
\par\addvspace{1.3em}


% That's all folks
\end{multicols*}

\end{document}
